\chapter{Description of a scene}
\label{model}

A scene in Move3D is an environement structure containing obstacles
and at least one robot. 

\section{Environment, robots and obstacles}

Two generic functions of the description of a Move3D scene are  {\tt
void *p3d\_beg\_desc(int type, char *name)} \index{p3d\_beg\_desc}
that starts the description of the element {\tt name} of type {\tt
type} (environement, robot, obstacle, body of a robot), and {\tt int
p3d\_end\_desc(void)} \index{p3d\_end\_desc} that ends the current
description.

\subsection{Description of an environment}

\index{P3D\_ENV}

The desciption of a scene always starts by function {\tt
p3d\_beg\_desc(P3D\_ENV,env\_name)}, that calls the function {\tt void
*p3d\_beg\_env(char* name)} \index{p3d\_beg\_env} with the name {\tt env\_name} as input. The
function {\tt p3d\_beg\_env} allocates an environment structure {\tt
p3d\_env} (cf Annex \ref{datastruct}) of name {\tt env},
intializes it and returns it as an output.

When all the robots and obstacles have been described, the function
{\tt p3d\_end\_desc()} is called, that calls the function {\tt int
p3d\_end\_env(void)} \index{p3d\_end\_env}. This function checks if some obstacles and at
least one robot have been described in the environment, and finishes
the initialization of the environnement.

Between the command {\tt p3d\_beg\_desc P3D\_ENV
env} and the corresponding command {\tt p3d\_end\_desc}, somes
obstacles and at least one robot must be described.

\subsection{Description of an obstacle}

\index{P3D\_OBSTACLE}

The description of an obstacle always start by the function {\tt
p3d\_beg\_desc(P3D\_OBSTACLE, obst\_name)}. This time, {\tt p3d\_beg\_desc}
checks if the description of an environment has been started, if we
are not describing a robot or another obtsacle at the same time and
calls the function {\tt void *p3d\_beg\_obj(char *name, int type)} \index{p3d\_beg\_obj}
with the name {\tt obst\_name} and the type {\tt P3D\_OBSTACLE} as inputs. 
The function {\tt p3d\_beg\_obj} allocates an object structure {\tt
p3d\_obj} (cf Annex \ref{datastruct}) of name {\tt obst\_name}, and
of type {\tt P3D\_OBSTACLE}, intializes it, stores it as the current
object of the environment and returns it as an output.

Once the obstacle has been initialized, the user can add one or several
polyhedrons to describe its geometry (cf Fig. \ref{FIG_ROBOT}). Those
polyhedrons can be created using regular geometric primitives or can
be described vertex by vertex.

\subsubsection{General polyhedron}

A general polyhedron description always starts by the function {\tt void
p3d\_add\_desc\_poly (char name[20])} \index{p3d\_add\_desc\_poly}
with the name of the polyhedron {\tt poly\_name} as an input. {\tt
p3d\_add\_desc \_poly} calls the function
{\tt p3d\_poly *p3d\_poly\_beg\_poly(char name[20])}
\index{p3d\_poly\_beg\_poly} with the name {\tt poly \_name}  as an
input and stores this new polyhedron as the current polyhedron of the
current object. The function {\tt p3d\_poly\_beg\_poly} allocates a
polyhedron  structure {\tt p3d\_poly} (cf Annex \ref{datastruct}) of
name {\tt poly\_name} , initializes it, and returns it as an output.

The vertices of the polyhedron {\tt poly\_name} are the described one by one,
using the function {\tt void p3d\_add\_desc\_vert(double x, double y, double
z)} \index{p3d\_add\_desc\_vert}. The function {\tt p3d\_add\_desc\_vert} calls the function {\tt
void p3d\_poly\_add\_vert(p3d\_poly *poly, float x, float y, float z)}
\index{p3d\_poly\_add\_vert}
with {\tt x, y, z} and the current polyhedron of the current object as
inputs. The function {\tt void p3d\_poly\_add\_vert} numbers the
vertex ({\tt x,y,z}), adds it to the current polyhedron and updates
its bounding box.

Once all the vertices have been described, we must describe the faces,
using the numbers of the vertices. A face description is a list of
integer {\tt a b c d ...} where the integers {\tt a,
b, c, d, ... } must be the numbers of the vertices of the face,
{\bf in the counter clockwise order if you face the exterior normal}
(cf Fig. \ref{FIG_FACE}).


\begin{figure}[hbt]
\centerline{
\psfig{figure=FIG/face.ps,width=7cm}
}
\caption{\small 
Description of a face of a general polyhedron
}
\label{FIG_FACE}
\end{figure}

Those numbers must be stored in an array of integer so the function {\tt void
p3d\_add \_desc\_face(int *listeV, int nb\_Vert)}
\index{p3d\_add\_desc\_face}can be called. The function {\tt
p3d\_add\_desc \_face} calls the function {\tt void p3d\_poly\_add\_face (p3d\_poly *poly, int *listeV, int nb\_Vert)} \index{p3d\_poly\_add\_face}
with the list of vertices of the face, the number of vertices in the
list and the current polyhedron of the current object as inputs. The
function {\tt p3d\_poly\_add\_face} builds the face of the current polyhedron.

Once all the faces have been described, the user must close the description
of the current polyhedron by the function {\tt void
p3d\_end\_desc\_poly(void)}\index{p3d\_end\_desc\_poly}. This
function calls the function {\tt void p3d\_poly\_end\_poly(p3d\_poly
*poly)} \index{p3d\_poly\_end\_poly}
with the current polyhedron as an input, udates the number of
polyhedron of the current object and its bounding box and places the
current polyhedron in the array of polyhedrons of the current
object. The function {\tt void p3d\_poly\_end\_poly} creates the edges of
the polyhedron and finishes its initialization.

This decription of a polyhedron can be very useful for a polyhedron of
specific or unusual shape, but most of the time we use regular
polyhedron. Move3d proposes a library of regular polyhedron called
primitives that simplify the description of polyhedrons.

\subsubsection{Geometric primitives}

The geometric primitives are all described in the same way : the
function {\tt void p3d \_add\_desc\_}{\it type}{\tt (char name[20],
}{\it parameters}{\tt )}
\index{p3d\_add\_desc\_type@p3d\_add\_desc\_{\it type}} is called,
{\it parameters} being the right parameters for the type of primitive
we choose (size of the side fo a cube, ray for a sphere,...).
The function {\tt p3d\_add\_desc\_}{\it type} calls the {\tt void
p3d\_create\_}{\it type}{\tt (char name[20], }{\it parameters}{\tt )}
\index{p3d\_create\_type@p3d\_create\_{\it type}}function with the
name {\tt prim} and the parameters as imputs, stores this new
polyhedron as the current polyhedron of the object, places it in the
array of polyhedrons of the current object and updates the number of
polyhedron of the current object and its bounding box. 

The different types of geometric primitives available in Move3D are
shown in Fig. \ref{FIG_PRIM}.\\


\begin{figure}[hbt]
\centerline{
\begin{tabular}{|c|c|c|c|c|}
\hline
\psfig{figure=FIG/cube.ps,height=2cm} &
\psfig{figure=FIG/box.ps, height=2cm} &
\psfig{figure=FIG/srect.ps,height=2cm} &
\psfig{figure=FIG/pyr.ps,height=2cm} &
\psfig{figure=FIG/cyl.ps,height=2cm} \\
\hline
cube & box & swept     & pyramid & cylinder\\ 
     &     & rectangle &         &         \\\hline
{\it cube} & {\it box} & {\it srect} & {\it pyramid} & {\it cylindre} \\\hline\hline
\psfig{figure=FIG/hoval.ps,height=2cm} &
\psfig{figure=FIG/pris.ps, height=2cm} &
\psfig{figure=FIG/snout.ps,height=2cm} &
\psfig{figure=FIG/ssnout.ps,height=2cm} &
\psfig{figure=FIG/cylov.ps,height=2cm} \\
\hline
half oval & prisma & snout & skew  & oval \\ 
  &        &       & snout & cylinder \\\hline
{\it half\_oval} & {\it prisme} & {\it snout} & {\it skew\_snout} & {\it cylindre\_oval} \\\hline\hline
\psfig{figure=FIG/rtorus.ps,height=2cm} &
\psfig{figure=FIG/cone.ps, height=2cm} &
\psfig{figure=FIG/sphere.ps,height=2cm} &
\psfig{figure=FIG/hsphere.ps,height=2cm} &
\psfig{figure=FIG/ctorus.ps,height=2cm} \\
\hline
rectangular  & cone & sphere & half   & circular \\ 
torus slice  &      &        & sphere & torus slice \\\hline
{\it rtorusslice} & {\it cone} & {\it sphere} & {\it half\_sphere} & {\it ctorusslice} \\\hline
\end{tabular}
}
\caption{\small 
The geometric primitives proposed in Move3D
}
\label{FIG_PRIM}
\end{figure}

A geometric primitive is always described with its center placed to
the point (0,0,0) in the frame of the environnement. So we often have
to put them in their right position, using the function {\tt void
p3d\_set\_prim\_pos(char name[20], double tx, double ty, double tz,
double rx, double ry, double rz)} \index{p3d\_set\_prim\_pos} that modify the position matrix of
the primitive. The parameter prim is the name of the primitive we want
to move, ({\tt tx, ty, tz}) the translations along the 3 axes, and
({\tt rx, ry, rz}) the rotations along the 3 axes (first {\tt rx} then
{\tt ry}  then {\tt rz}). This function can also be used for a general
polyhedon.

%p3d\_set\_prim\_pos\_by\_mat\\

Once all the polyhedron of the object have been described, the
function {\tt p3d\_end\_desc()} is called, that
calls this time the function {\tt int p3d\_end\_obj(void)} \index{p3d\_end\_obj}. This
function stores the current object in the array of obstacles of the
current environment if its type was {\tt P3D\_OBSTACLE}. 

After its description, a whole obstacle can be moved using the
function {\tt void p3d\_set \_obst\_pos(char name[20], double tx,
double ty, double tz, double rx, double ry, double rz)}
\index{p3d\_set\_obst\_pos} is called,
wich use is exactly the same than {\tt p3d\_set\_prim\_pos}.


%p3d\_set\_obst\_pos\_by\_mat\\


\subsection{Description of a robot}

\index{P3D\_ROBOT}

A robot is made of one or several bodies linked by some joints. In
Move3D, a body is the same kind of object than an obstacle. The
description of a robot always starts by the function {\tt
p3d\_beg\_desc(P3D\_ROBOT, robot\_name)}. The function {\tt p3d\_beg\_desc}
checks again if the description of an environment has been started, if we
are not describing an obstacle or another robot at the same time and
calls the function {\tt void *p3d\_beg\_rob(char *name)}
\index{p3d\_beg\_rob} with the name
{\tt robot\_name}. The function {\tt p3d\_beg\_rob} allocates an robot structure {\tt
p3d\_rob} (cf Annex \ref{datastruct}) of name {\tt robot}, initializes
it stores it as the current robot, and returns it as an output. 

Once the robot has been initialized, we can start its geometrical and
cinematical description.

A robot can be seen as a kinematic chain (a tree) on which some polyhedral bodies are
placed (cf Fig. \ref{FIG_ROBOT}) but it can also be seen as a ``game
of construction'' : we create a body, we attach a joint on this body
and then place another body on this joint, and so on.

\index{P3D\_BODY}
The description of a body starts by the function {\tt
p3d\_beg\_desc(P3D\_BODY,body\_name)}. The function {\tt p3d\_beg\_desc}
checks if the description of an environment and of a robot have been
started, if we are not describing another robot or obtsacle at the
same time, and, as a body is the same type of object as an obstacle,
it calls the function {\tt void *p3d\_beg\_obj(char *name, int type)}
with the name {\tt body\_name} and the type {\tt P3D\_BODY} as inputs. The
description of a body of a robot is the same as the description of an
obstacle, but with some slight differences.

Inside {\tt p3d\_beg\_obj}, there is no difference : the object is
allocated, initialized and stored as the current object. But inside
{\tt p3d\_end\_obj}, the current object is not stored in the array of
obstacles of teh environment, but in the array of bodies of the robot.
If the body described is the first body of the robot, {\tt
p3d\_end\_obj} also creates automatically the joint 0 of the robot,
j0. The joint j0 is a special joint : it allows the robot to translate
along the $x$, $y$ and $z$ axis and to rotate around the $z$ axis. It
is always defined at the point $(0,0,0)$. And this induces the main
difference between the description of an obstacle and the description
of a robot : an obstacle can be described in his place in the
environnement but {\bf a robot must be described regarding the point
$(0,0,0)$}. 
The description of a body ends with a {\tt p3d\_end\_desc}. 

A joint is added with the function :

\noindent{{\tt int p3d\_add\_desc\_jnt(int type, double x0, double y0, double
z0, double axe1, double axe2, double axe3, double v, double vmin,
double vmax, int prev)}}

\index{p3d\_add\_desc\_jnt}

The type {\it TYPE} can be {\tt P3D\_ROTATE} \index{P3D\_ROTATE} (for a rotoid joint) or
{\tt P3D\_TRANLSATE} \index{P3D\_TRANLSATE} (for a prismatic joint). The triplet $(x0,y0,z0)$ is
the point where the two bodies will be linked and the vectore
$(axe1,axe2,axe3)$ is the axis of rotation or translation of the
joint. The interval $[vmin,vmax]$ is the bounds of the joint and $v$ its
current value. Finally, $prev\_jnt$ is the number of the previous
joint of the described joint on the cinematic chain.

It checks if a description of a robot has been started, if we are not
describing an obstacle at the same time, allocate the joint,
initializes it and updates the informations of the cinematic chain
(the set of next joints of the rpevious joint for example), stores
it in the array of joints of the robot, and returns the new number of
joints of the robot.

The description of a robot ends by a {\tt p3d\_end\_desc} that calls
the function {\tt int p3d\_end\_rob(void)} \index{p3d\_end\_rob}. The function {\tt
p3d\_end\_rob} updates the bounding box of the robot, and finishes the
initialization.

%p3d\_set\_body\_pos\_by\_mat\\



\begin{figure}[hbt]
\centerline{
\psfig{figure=/home/nissoux/LATEX/THESE/CHAPIII/FIG/modele_hil.ps,width=10cm}
}
\caption{\small 
Description of a scene in Move3D.
}
\label{FIG_ROBOT}
\end{figure}

Once all the obstacles and robots have been described, the description
of a scen ends by a last {\tt p3d\_end\_desc}. But to be complete, a
scene must also use the functions {\tt p3d\_set\_env\_box} (defines the
limits of the scene), {\tt p3d\_set\_robot\_box} (defines the bounds
of j0), {\tt p3d\_set\_robot\_radius} (define the radius of the
turning circle of the robot) and {\tt p3d\_set\_robot\_pos} (defines
the current values of the degrees of freedom of j0) that will be
described in the following chapters.

\section{Trajectories}

In Move3D, the trajectory is described by a sequence of elementary
curves (typically straight line segments in the configuration space,
arcs of circle for trajectories of car-like robots,...).

The description of a trajectory always starts by the command {\tt
p3d\_beg\_desc P3D\_TRAJ traj}, that checks checks again if the
description of an environment has been started, if we are not
describing another trajectory at the same time and calls the
function {\tt void *p3d\_beg\_traj(char *name)} \index{p3d\_beg\_traj}
with the name {\tt
traj}. The function {\tt p3d\_beg\_traj} allocates an object structure
{\tt p3d\_trj} (cf Annex \ref{datastruct}) of name {\tt traj},
intializes it, stores it as the current trajectory of the environnemnt
and returns it as an output.

The function {\tt int p3d\_add\_desc\_courbe(p3d\_courbe *cadd,int
nmaill)} \index{p3d\_add\_desc\_courbe} adds to the current trajectory the sequence of {\tt nmaill}
elementary curves of type {\tt p3d\_courbe} (cf Annex
\ref{datastruct}) contained in the list {\tt cadd} (typically a local
path) . and returns the current number of elementary curves of the
current trajectory.

The function {\tt int p3d\_end\_traj(void)} \index{p3d\_end\_traj} stores the current
trajectory in the array of trajectories of the current robot.


\section{Deleting a scene}

The function {\tt int p3d\_del\_desc(int type)} \index{p3d\_del\_desc}
deletes the current element of type {\tt type}. The function {\tt
p3d\_del\_desc} calls the functions {\tt int p3d\_del\_obst(pp3d\_obj
o)}, {\tt int p3d\_del\_rob(pp3d\_rob r)}, {\tt int
p3d\_del\_traj(pp3d\_trj t)}, or {\tt int p3d\_del\_env(pp3d \_env e)}
regarding the type of the element the user wants to delete. 

The function {\tt int p3d\_del\_traj(pp3d\_trj t)}
\index{p3d\_del\_traj} liberates the
trajectory {\tt t}, the elementary curves of {\tt t} and actualizes
the array of trajectories of the robot corresponding to {\tt t}.

The function {\tt int p3d\_del\_obst(pp3d\_obj o)}
\index{p3d\_del\_obst} liberates the
object {\tt o}, the polyhedrons of {\tt o} (with the function
p3d\_poly\_del\_poly) and modifies the current obstacle if this one
was {\tt o}).

The function {\tt int p3d\_del\_rob(pp3d\_rob r)}
\index{p3d\_del\_rob} liberates the robot
{\tt r}, its joints, its bodies, its trajectories, its initial and
final configuration, actualizes the array of robots of the environment
and modifies the current robot if this one was {\tt r}. 

The function {\tt int p3d\_del\_env(pp3d\_env e)}
\index{p3d\_del\_env} liberates the
environment {\tt e}, liberates all its obstacles and robots and
actualizes the array of environments of the Move3D sessions.

The function {\tt int p3d\_poly\_del\_poly(p3d\_poly *p)}
\index{p3d\_poly\_del\_poly} liberates
the polyhedron {\tt p} and actualizes the set of poyhedrons of the
scene.



