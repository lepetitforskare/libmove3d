\chapter{Steering methods}

\section{Building a local path}

A steering method builds a path between two configurations of a robot,
respecting its kinematic constraints, but without considering any
obstacle. In Move3D, a steering method is a function that takes a
robot {\tt r} and two configurations {\tt qi} and {\tt qf} as inputs
and returns a list of {\tt nmaill} elementary curves {\tt c}. Move3D
has five different steering methods. 

The function {\tt p3d\_courbe *p3d\_linear\_search(p3d\_rob *r, double
*qi, double *qf, int *nmaill)} \index{p3d\_linear\_search} checks
wether {\tt qi} and {\tt qf} are equal, if they are free, and builds
an elementary curve containing the straight line segment in the
configuration space between {\tt qi} and {\tt qf} (cf
Fig. \ref{FIG_LOCAL_SEG}). In this function, nmaill is always equal to
1.

\begin{figure}[hbt]
\centerline{
\psfig{figure=FIG/localseg.ps,width=5cm}
}
\caption{\small 
Example of local path built by {\tt p3d\_linear\_search}.
}
\label{FIG_LOCAL_SEG}
\end{figure}  

The function {\tt p3d\_courbe *p3d\_local\_search(p3d\_rob *r, double
*qi, double *qf, int *nmaill)} \index{p3d\_local\_search} checks wether {\tt qi} and {\tt qf} are
equal, if they are free, and builds the elementary curves containing
the Reeds and Shepp path for the $x,y,\theta$ degrees of freedom of
{\tt qi} and {\tt qf} (the first, second and fourth ones) and the
straight line segment in the configuration space for the others (cf
Fig. \ref{FIG_LOCAL_RS_SEG}). In this function, nmaill can range from 3 to 5.

\begin{figure}[hbt]
\centerline{
\psfig{figure=FIG/localRSseg.ps,width=5cm}
}
\caption{\small 
Example of local path built by {\tt p3d\_local\_search}.
}
\label{FIG_LOCAL_RS_SEG}
\end{figure}

The function {\tt p3d\_courbe *p3d\_plateform\_search(p3d\_rob *r,
double *qi, double *qf, int *nmaill)} \index{p3d\_plateform\_search}
checks wether {\tt qi} and {\tt qf} are equal, if they are free, and
builds the elementary curves containing the Reeds and Shepp path for
the $x,y,\theta$ degrees of freedom of {\tt qi} and {\tt qf}, and
blocks the others in {\tt qi} (cf Fig. \ref{FIG_LOCAL_RS}). In this
function, nmaill can also range from 3 to 5.

\begin{figure}[hbt]
\centerline{
\psfig{figure=FIG/localRS.ps,width=5cm}
}
\caption{\small 
Example of local path built by {\tt p3d\_platform\_search}.
}
\label{FIG_LOCAL_RS}
\end{figure}  

The function {\tt p3d\_courbe *p3d\_arm\_search(p3d\_rob *r, double
*qi, double *qf, int *nmaill)} \index{p3d\_arm\_search} checks wether
{\tt qi} and {\tt qf} are equal, if they are free, and builds an
elementary curve containing the straight line segment in the
configuration space between {\tt qi} and {\tt qf} except for the
degrees of freedom $x,y,\theta$ that are blocked in their {\tt qi}
values. In this function, nmaill is always equal to 1.

The function {\tt p3d\_courbe *p3d\_manhattan\_search(p3d\_rob *r,
double *qi, double *qf, int *nmaill)} \index{p3d\_manhattan\_search}
checks wether {\tt qi} and {\tt
qf} are equal, if they are free, and builds the elementary curves
containing a symmetric Manhattan local path. This paths is made of
straight line segments representing the move of a single degree of
freedom at the same time. If the first degree of freedom $x$ of {\tt
qi} is smaller than the first degree of freedom of {\tt qf} we move
the first degrees of freedom first, and if the first degree of freedom
of {\tt qi} is greater than the first degree of freedom of {\tt qf},
we move the last degrees of freedom first, so that the local path is
symmetrical(cf Fig. \ref{FIG_LOCAL_MAN}).In this function, nmaill is always
equal to the number of degrees of freedom.

\begin{figure}[hbt]
\centerline{
\psfig{figure=FIG/localman.ps,width=5cm}
}
\caption{\small 
Example of local path built by {\tt p3d\_manhattan\_search}.
}
\label{FIG_LOCAL_MAN}
\end{figure}

\section{Validating a local path}

Validating a local paths is knowing if, for a given robot
corresponding to this local path, the robot will collide moving along
this path. The Move3d validation of local path is simple : Being given
the local path of a robot, we place this robot at the begining of the
path and compute dmin, the closest distance of the bounding boxes of
the bodies of the robot to the bounding boxes of the obstacles. This
gives us the increment of configuration dq along which the robot can
move without that any of its point moves further than dmin. So we can
be sure that there will not be any collision along dq. If the bounding
boxe of a body overlaps the bounding box of an obstacle, dq becomes
regular and very small so we can be sure to detect a collision between
the body and the obstacle if there is one (cf Fig \ref{FIG_VALID}).


\begin{figure}[hbt]
\centerline{
\psfig{figure=FIG/valid.ps,width=10cm}
}
\caption{\small 
Validation of a local path in Move3D.
}
\label{FIG_VALID}
\end{figure}  

The function {\tt int p3d\_col\_test\_traj\_dist(p3d\_rob *rob,
p3d\_courbe *c, int nmaill, int *ntest, int *nbb)}
\index{p3d\_col\_test\_traj\_dist} test if the robot
{\tt rob} collides while moving along the local path made of the {\tt
nmaill} elementary curves contained in the list {\tt c}. The integer
{\tt ntest} is the number of calls to the collision checker, and {\tt
nbb} the number of times we have to check the distance between the
bounding boxes.



