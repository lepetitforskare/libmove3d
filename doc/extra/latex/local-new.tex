\def\hsp{\hspace*{1cm}}

\chapter{Steering methods and local paths}
\label{methode-local}
\def\CS{{\cal C}}

Local path is the element that encodes elementary motion in Move3D:
trajectories are concatenations of local paths, edges of roadmaps
contain local paths. A local path for a given robot is produced by a
local method. In this chapter, we first give a definition of what we
mean by local path. Then we show how different types of local paths
can be handled by Move3D. Finally, we describe the different steering
methods implemented within the software.

\section{Definition of local path and steering method}

\paragraph{Definition and parameterizations of local path.}
A {\em local path} for a robot $r$ is a continuous curve in the
configuration space of the robot and defined over an interval $[0,U]$:
$$
\begin{array}{lcll}
lp: & [0,U] & \rightarrow & \CS \\
    & u &\rightarrow & lp(u)
\end{array}
$$
In the rest of this chapter, $u$ is called the default parameter (or
parameter) of the local path $lp$ and $U$ the parameter range of $lp$.

 The curve can also be parameterized by length (or
  by pseudo-length), 

In addition to the default parameterization, a length parameterization
can be defined. This parameterization has to be defined by the
developper.  The length parameterization is mainly used for optimizing
trajectories.  The optimizing function picks random configurations
over a list of successive local paths. The length parameterization
ensures us that longer local paths are more likely to be chosen than
smaller ones. The length parameterization can be the same as the
default parameterization u.

The length parameter is defined from the default parameter $u$ by an
increasing function $d$ from $[0,U]$ to an interval $[0,L]$. $L=d(U)$
is called the length of the local path. It does not necessarily represent 
a real length. 

\paragraph{Steering method.}
A steering method is a function that maps, to a pair of configurations, a
local path between these configurations. The simplest steering method is the 
linear one. It connects the two configurations by a  straight line in the
space of configuration parameters.
Some robots are subjected to kinematic constraints such as rolling
without slipping for wheeled mobile robots or moving only one degree
of freedom at a time for a rolling bridge. These kinematic constraints
are not specified in the description of the mechanical system. They
are taken into account by the local method. For instance Reeds and Shepp 
steering method builds curves composed of straight lines and arc of circles
for a carlike robot.

\section{Representation of a local path}
\label{sec:representation}

Local paths are encoded by the data-structure {\tt
  p3d\_localpath}\index{Data structures!p3d\_localpath}

A local path is encoded by a variable of type {\tt p3d\_localpath}. This
structure contains 3 types of fields:
\begin{itemize}
\item data generic to any local path,
\item a pointer to data specific to the type of local path
\item pointers to functions specific to the type of local path.
\end{itemize}

The structure {\tt p3d\_localpath} is like an abstract class in an
object-oriented langage. The specific functions are methods associated
to each derived class (each type of local path).

We enumerate now the main fields of the structure {p3d\_localpath}.

\subsubsection*{data generic to any local path.}
\begin{itemize}
\item {\tt type\_lp}: type of local path (linear, reeds and shepp,...).
\item {\tt range\_param}: parameter range of the local path.
\item {\tt length\_lp}: range of the length parameterization.
\end{itemize}

\subsubsection*{specific methods associated to a local path}
\begin{itemize}
\item {\tt length}: computes the length of the local path.
\item {\tt copy}: copies the local path.
\item {\tt extract}: extracts from a given local path the part included between two values of the length parameter of the local path.
\item {\tt destroy}: destroyes the local path.
\item {\tt config\_at\_distance}: computes the configuration at a given
  distance along the local path.
\item {\tt config\_at\_param}: computes the configuration at a given
  parameter along the local path.
\item {\tt stay\_within\_dist}: given the distance of each body of the
  robot to the obstacles, and a position on a local path, this
  function computes an interval of default parameter that ensures the
  user that each body does not move by more than its specified
  distance when the parameter stays in this interval.
\item {\tt cost}: computes the cost of a local path. This cost is used
  in $A^{*}$ procedure to find the best path in the roadmap and during
  optimization to replace a part of a path by a new local path if this
  latter has lower cost.
\item {\tt simplify}: This function does nothing for most local
  paths. given two successive local paths. When two Reeds and Shepp
  local paths are connected, the end of the first one may overlap the
  beginning of the second one. This function detects this kind of
  situation and removes common parts.
\end{itemize}

Data specific to each type of local path are detailled in the
following section.

\section{Local methods implemented within Move3D}

In this section, we describe the various local methods currently supported 
by Move3D. For each of them, we specify the default and length parameterization
used and the data-structures involved.

\subsection{Linear local method}
\label{subsec:linear}

This local method is the simplest one. It connects two configurations
by a straight line in the joint parameter space. If a configuration
parameter represents the rotation angle of a joint without bounds,
along a linear path, this parameter follows the shortest arc between
its initial and final values.

\begin{figure}[ht]
\centerline{\psfig{figure=FIG/localseg.ps, width=6cm}}
\caption{Example of linear local path}
\label{fig:linear-local}
\end{figure}

\subsubsection*{parameterization}

For this type of local path the default and length parameterizations
are the same. Given two configurations, $q_0=(x_0,y_0,z_0,\theta_0,
\phi^{1}_{0},...,\phi^{n}_{0})$ and $q_1=(x_1,y_1,z_1,\theta_1,
\phi^{1}_{1},...,\phi^{n}_{1})$ where the $\phi^i$'s represent the values 
of each joint, we define a distance as follows:
$$
d_{lin}(q_0,q_1) = \sqrt{(x_1-x_0)^2 + (y_1-y_0)^2 + (z_1-z_0)^2 + d_{S^1}(\theta_1-\theta_0)^2 + \sum_{i=1}^{n} d_i(\phi^i_0,\phi^i_1)^2}
$$
where $d_{S^1}$ is the arc length distance over the unit circle, 
\begin{itemize}
\item $d_i(\phi^i_0,\phi^i_1) = |\phi^i_1-\phi^i_0|$ if joint $i$ is a
  translation joint,
  
\item $d_i(\phi^i_0,\phi^i_1) = dist\,|\phi^i_1-\phi^i_0|$ where
  $dist$ is the maximal distance between the points of body $i$ and
  the axis of joint $i$, if this latter is a rotation joint with
  bounds.

\item $d_i(\phi^i_0,\phi^i_1) = dist\, d_{S^1}(\phi^i_0,\phi^i_1)$ if
  joint $i$ is a rotation joint without bounds (i.e. that can rotate
  freely about its axis).
\end{itemize}
The factor $dist$ in front of angular distances makes the linear
distance over the configuration space homogeneous to a length.

\subsubsection*{Data-structure}

Data relative to a linear local path are stored in the following
structure:
\index{Data structures!struct lin\_data}
\index{Data structures!p3d\_lin\_data}
\begin{tabular}{l}{\tt 
struct lin\_data} $\{$ \\
\hsp {\tt configPt q\_init;} \\
\hsp {\tt configPt q\_end;} \\
$\}$ {\tt p3d\_lin\_data, *pp3d\_lin\_data;}\\
\end{tabular}

\noindent
where {\tt q\_init} and {\tt q\_end} are the initial and final
configurations of the local path.

\subsection{Reeds and Shepp + Linear local method}

This local method connects two configurations by Reeds and Shepp
curves in the plane $(x,y)$ for the platform of the robot and linear
curves for the other joints: given two configurations, 
$q_0=(x_0,y_0,z_0,\theta_0, \phi^{1}_{0},...,\phi^{n}_{0})$ and 
$q_1=(x_1,y_1,z_1,\theta_1,\phi^{1}_{1},...,\phi^{n}_{1})$,
$(x_0,y_0,\theta_0)$ and $(x_1,y_1,\theta_1)$ are connected by a
Reeds and Shepp curve of length
$l_{RS}(q_0,q_1)$. $(z_0,\phi^{1}_{0},...,\phi^{n}_{0})$ and
$(z_1,\phi^{1}_{1},...,\phi^{n}_{1})$ are connected
using the linear strategy described in the previous section.

\begin{figure}[ht]
\centerline{\psfig{figure=FIG/RS.ps, width=6cm}}
\caption{Example of Reeds and Shepp + linear local path}
\label{fig:RS-local}
\end{figure}

\subsubsection*{Parameterization}

\paragraph{Default parameterization:} the default parameterization is
the arc-length distance along Reeds and Shepp curve. Thus, the
parameter range is $l_{RS}(q_0,q_1)$.

\paragraph{Length parameterization:} the distance between
configurations $q_0$ and $q_1$ is defined as follows:
$$
d_{RS-lin}(q_0,q_1) = l_{RS}(q_0,q_1) + \sqrt{\sum_{i=1}^{n} d_i(\phi^i_0,\phi^i_1)^2}
$$
where distances $d_i$ are defined as in the previous section. Notice
that the coordinate $z$ does not appear in this distance.

\subsubsection*{Data-structure}

Data relative to a Reeds and Shepp + linear local path are stored in
a list of the following structure:

\index{Data structures!struct rs\_data}
\index{Data structures!p3d\_rs\_data}

\begin{tabular}{l}
{\tt struct rs\_data} $\{$ \\
\hsp {\tt  configPt q\_init;} \\
\hsp {\tt  configPt q\_end;} \\
\hsp {\tt  double centre\_x;} \\
\hsp {\tt  double centre\_y;} \\
\hsp {\tt  double radius;} \\
\hsp {\tt  whichway dir\_rs;} \\
\hsp {\tt  double val\_rs;} \\
\hsp {\tt  rs\_type type\_rs;} \\
\hsp {\tt  struct rs\_data *next\_rs;} \\
\hsp {\tt  struct rs\_data *prev\_rs;} \\
$\}$ {\tt p3d\_rs\_data, *pp3d\_rs\_data;} \\
\end{tabular}

\noindent
each of these structure represent a Reeds and Shepp segment, that is
either a straight line or a left or right turn. 
\begin{itemize}
\item {\tt q\_init} and {\tt q\_end} are the initial and final
  configurations of the RS segment. 
\item {\tt centre\_x}, {\tt centre\_y} is the center of rotation in
case of a right of left turn, $(0,0)$ in case of a straight line. 
\item {\tt radius} is the radius of curvature 
\item {\tt dir\_rs} is the direction (forward or backward) of motion 
\item {\tt val\_rs} is the arc length of the RS segment. 
\item {\tt type\_rs} is the type of segment (right, left turn or
  straight line)
\item {\tt next\_rs} and {\tt prev\_rs} are pointers to the next and
  previous RS segment in the local path.
\end{itemize}

\subsection{Manhattan local method}

A manhattan local path is a path along which only one coordinate moves
at a time. 
Given two configurations, $q_0=(x_0,y_0,z_0,\theta_0,
\phi^{1}_{0},...,\phi^{n}_{0})$ and
$q_1=(x_1,y_1,z_1,\theta_1,\phi^{1}_{1},...,\phi^{n}_{1})$, the
Manhattan local path between them is the concatenation of the
following sub-paths.
\[
\begin{array}{cc}
((1-\alpha)x_0+\alpha x_1,y_0,z_0,\theta_0,\phi^{1}_{0},...,\phi^{n}_{0}) &
0 \leq \alpha \leq 1 \\
(x_1,(1-\alpha)y_0+\alpha y_1,z_0,\theta_0,\phi^{1}_{0},...,\phi^{n}_{0}) &
0 \leq \alpha \leq 1 \\
\vdots & \vdots \\
(x_1,y_1,z_1,\theta_1,\phi^{1}_{1},..., (1-\alpha)\phi^{n}_{0}+\alpha\phi^{n}_{1})& 0 \leq \alpha \leq 1
\end{array}
\]
where the interpolation $(1-\alpha)\phi^{n}_{0}+\alpha\phi^{n}_{1}$
between two angles follows again the shortest arc when possible.
To make the method symmetric, that is, to make it follow the same path
between $q_0$ and $q_1$ as between $q_1$ and $q_0$, we either move the
coordinates in order $x$ to $\phi^{n}$ or in order $\phi^{n}$ to $x$
according to the following criterion:
\begin{itemize}
\item if $x_0<x_1$, move coordinates from $x$ to $\phi^{n}$,
\item if $x_1<x_0$, move coordinates from $\phi^{n}$ to $x$,
\item if $x_1=x_0$, apply criterion to $y_0$ and $y_1$ or to the
  first coordinate different in $q_0$ and $q_1$.
\end{itemize}

\subsubsection*{parameterization}

As for linear local paths, the default and length parameterizations
are the same. Given two configurations, $q_0=(x_0,y_0,z_0,\theta_0,
\phi^{1}_{0},...,\phi^{n}_{0})$ and $q_1=(x_1,y_1,z_1,\theta_1,
\phi^{1}_{1},...,\phi^{n}_{1})$ where the $\phi^i$'s represent the values 
of each joint, we define the Manhattan distance as follows:
$$
d_{manh}(q_0,q_1) = |x_1-x_0| + |y_1-y_0| + |z_1-z_0| + d_{S^1}(\theta_1-\theta_0) + \sum_{i=1}^{n} d_i(\phi^i_0,\phi^i_1)
$$
where $d_{S^1}$ and the $d_i$ are defined in Section~\ref{subsec:linear}.

\subsubsection*{Data-structure}

Data relative to a Manhattan local path are stored in the following
structure:

\index{Data structures!struct manh\_data}
\index{Data structures!p3d\_manh\_data}
\begin{tabular}{l}{\tt 
struct manh\_data} $\{$ \\
\hsp {\tt configPt q\_init;} \\
\hsp {\tt configPt q\_end;} \\
$\}$ {\tt p3d\_manh\_data, *pp3d\_manh\_data;}\\
\end{tabular}

\noindent
where {\tt q\_init} and {\tt q\_end} are the initial and final
configurations of the local path.



\subsection{Trailer local method}

This local method takes into account the nonholonomic constraints of a
robot towing a trailer. Such a system is represented on
figure~\ref{fig:trailer-robot}. 
\begin{figure}[ht]
\centerline{\epsfig{figure=FIG/trailer-robot.eps,width=5cm}}
\caption{A robot with a trailer}
\label{fig:trailer-robot}
\end{figure}
$a$ and $b$ denote the distances between the connection point and
respectively the robot wheel axis and the trailer wheel axis.

The trailer local method uses the property of differential flatness of
the tractor trailer system to build feasible paths between any pair
of configurations of the system. Details about this local method can
be found in~\cite{these-sepanta,these-florent}.

A local path built by this local method is composed of either one
smooth motion or two smooth motions of opposite direction
(backward-forward or forward-backward).

\begin{figure}[ht]
\centerline{\psfig{figure=FIG/trailer.ps, width=6cm}}
\caption{Example of local path computed by the trailer steering method}
\label{fig:trailer-local}
\end{figure}

\subsubsection*{parameterization}

The default parameterization goes from 0 to 1 if the local path is
composed of one part and from 0 to 2 if the local path is composed of
two parts. The length parameterization is proportional to the default
parameterization. The length of a local path is defined
in~\cite{rapport-elodie}. 

\subsubsection*{Data-structure}

Data relative to a trailer local path are stored in the following
structure:

\index{Data structures!struct trailer\_data}
\index{Data structures!p3d\_trailer\_data}
\begin{tabular}{l}
{\tt struct trailer\_data} $\{$ \\
\hsp{\tt   p3d\_sub\_trailer\_data *init\_cusp;} \\
\hsp{\tt   p3d\_sub\_trailer\_data *cusp\_end;} \\
$\}$ {\tt  p3d\_trailer\_data, *pp3d\_trailer\_data;} \\
\end{tabular}

\noindent
where {\tt init\_cusp} and {\tt cusp\_end} are pointers to sub-local
paths of the type convex combination of canonical
curves~\cite{these-florent}. We do not give more details here about how
these sub-local paths are encoded. It would require further
developments that are out of the scope of this document.

\subsubsection*{Motion execution aspects}

The trailer local method is used at LAAS to plan paths for Hilare
towing a trailer. To follow a series of local paths returned by the
local method without stopping between two local paths, the curvature
of the curve followed by the center of the robot has to be continuous.
To make this curvature continuous along a path, we have to store it in
a configuration variable. For that, we define in the configuration
file of the robot a virtual joint attached to the robot and that
encodes this curvature\footnote{In fact, the value of the
  configuration variable of the virtual joint is the derivative of the
  curvature of the flat output w.r.t. an arc-length
  parameterization.}. Figure~\ref{fig:virtual-joint} shows an example
of configuration file for Hilare and its trailer.

\begin{figure}[ht]
\begin{tabular}{l}
{\tt p3d\_beg\_desc P3D\_ROBOT } \\
{\tt p3d\_beg\_desc P3D\_BODY body } \\
\hsp\hsp\hsp $\vdots$ \\
{\tt p3d\_end\_desc} \\
{\tt } \\
{\tt \# derivative of the curvature (-3000 < dK/ds < 3000)} \\
{\tt p3d\_add\_desc\_jnt P3D\_TRANSLATE 0 0 0  0 0 1 0.0 -3000.0 3000.0 0
  0 0} \\
{\tt } \\
{\tt \# angle between the robot and the trailer (-70 deg < phi < 70 deg)} \\
{\tt p3d\_add\_desc\_jnt P3D\_ROTATE -0.65 0 0.92 0 0 1 0.0 -70.0 70.0 0} \\
{\tt } \\
{\tt p3d\_beg\_desc P3D\_BODY trailer} \\
\hsp\hsp\hsp $\vdots$ \\
{\tt p3d\_end\_desc} \\
{\tt p3d\_end\_desc} \\
\end{tabular}
\caption{Example of configuration file for Hilare and its trailer
  ({\tt hilare-trailer.macro}). The first joint is the virtual joint.
  The bounds $[-3000,3000]$ can be set to any value, they are
  recomputed by Move3D. The second joint corresponds to the connection
  between the trailer and the robot.}
\label{fig:virtual-joint}
\end{figure}

\subsubsection*{local method parameters}

Applying the trailer local method to a given robot requires to
initialize some geometric parameters. These parameters are defined in
the configuration file describing the environment and are stored in the
data-structure that describes the robot. These geometric parameters
are the following~:
\begin{itemize}
\item the connection length $a$ and $b$
\item the joint Id corresponding to the trailer connection on the
  robot
\item the virtual joint Id.
\end{itemize}

These geometric parameters are set for each robot in the configuration
file of the scene by the following command~:

\begin{tabular}{l}
{\tt p3d\_sel\_desc\_name P3D\_ROBOT robot\_name} \\
{\tt p3d\_set\_local\_method\_params a b j1 j2}\index{p3d\_set\_local\_method\_params}
 \\
\end{tabular}

where {\tt robot\_name} is the name of the robot to initialize, $a$
and $b$ are the trailer connection lengths defined on
Figure~\ref{fig:trailer-robot}, $j1$ and $j2$ are the joint Id's
corresponding to respectively the trailer connection and the virtual
joint. Let us recall that joints are numbered increasingly from 1 in
the order they are defined. Joint 0 corresponds to the main body of
the robot. Figure~\ref{fig:local-method-params} shows an example of a
scene containing the type of robot described in
Figure~\ref{fig:virtual-joint}. 

\begin{figure}[ht]
{\tt p3d\_beg\_desc P3D\_ENV robotics-room} \\
{\tt p3d\_read\_macro hilare-trailer.macro trailer-robot} \\
\hsp\hsp\hsp $\vdots$ \\
{\tt p3d\_sel\_desc\_name P3D\_ROBOT trailer-robot} \\
{\tt p3d\_set\_local\_method\_params 0.65 0.90 2 1} \\

\caption{Initialization of the geometric parameters of trailer local
  method.}
\label{fig:local-method-params}
\end{figure}

\subsubsection*{The car modeled as a robot-trailer system}

The trailer local method can also plan paths for a car by modelling
the car as a robot-trailer system as shown on
Figure~\ref{fig:car-model}. The angle of the wheel axis depends on the
angle between the virtual robot and the car body. This dependence is
modeled by a kinematic constraint as described in Chapter~\ref{Constraints}
\begin{figure}[ht]
\centerline{\epsfig{figure=FIG/car-model.eps,width=5cm}}
\caption{The car modeled as a robot with trailer: the robot is a virtual
  body and the body of the car is the trailer. The front wheels are
  connected to the body of the car by vertical axis rotation
  joints. $a=0$ and $b$ is the distance between the rear axis and the
  front wheels.}
\label{fig:car-model}
\end{figure}

\section{How to implement a new local method within Move3D}

To implement a new local planner, the user has first to define his own
structure specific local path and then to implement functions specific
to this type of local path. In the following sections, we describe the
different actions to perform to define a new type of local path and
steering method.

\subsection{Definition of a new local path type}

In the file {\tt include/localpath.h}, define a new structure with the
data relative to the new type of local path.

\begin{tabular}{l}
{\tt typedef struct newlp\_data} $\{$ \\
\hsp\hsp\hsp$\vdots$ \\
$\}$ {\tt p3d\_newlp\_data, *pp3d\_newlp\_data;} \\
\end{tabular}

\noindent
Add this structure to the set of possible specific data of a local path

\index{Data structures!union lm\_specific}
\index{Data structures!p3d\_lm\_specific}
\begin{tabular}{l}
{\tt typedef union lm\_specific} $\{$\\
\hsp{\tt   pp3d\_rs\_data rs\_data;}\\
\hsp{\tt   pp3d\_lin\_data lin\_data;}\\
\hsp{\tt   pp3d\_manh\_data manh\_data;}\\
\hsp{\tt   pp3d\_trailer\_data manh\_data;}\\
\hsp{\tt   pp3d\_newlp\_data newlp\_data;}\\
$\}$ {\tt p3d\_lm\_specific, *pp3d\_lm\_specific;}\\
\end{tabular}

\noindent
Add the new type of local paths in the set of types of local paths

\index{p3d\_localpath\_type}
\begin{tabular}{l}
{\tt typedef enum } $\{$\\
\hsp{\tt   REEDS\_SHEPP,}\\
\hsp{\tt   LINEAR,}\\
\hsp{\tt   MANHATTAN,}\\
\hsp{\tt   TRAILER,}\\
\hsp{\tt   NEWLP,}\\
$\}${\tt p3d\_localpath\_type;}\\
\end{tabular}

\noindent
Add a local planner in the list, to produce the new type of local paths

\index{p3d\_localplanner\_type}
\begin{tabular}{l}
{\tt typedef enum }  $\{$\\
\hsp{\tt   P3D\_RSARM\_PLANNER,}\\
\hsp{\tt   P3D\_LINEAR\_PLANNER,}\\
\hsp{\tt   P3D\_MANH\_PLANNER,}\\
\hsp{\tt   P3D\_TRAILER\_PLANNER,}\\
\hsp{\tt   P3D\_NEWLP\_PLANNER,}\\
$\}${\tt  p3d\_localplanner\_type;}\\
\end{tabular}

\subsection{Definition of a new local planner}

In the file {\tt localpath/p3d\_local.c}

\noindent
Add a function calling the local planner in {\tt array\_localplanner}

\begin{tabular}{l}
\index{ptr\_to\_localplanner}
\index{array\_localplanner}
{\tt ptr\_to\_localplanner array\_localplanner[]= }\\
$\{$ \\
\hsp{\tt   (pp3d\_localpath (*)(p3d\_rob*, configPt,
  configPt))(p3d\_rsarm\_localplanner), }\\
\hsp{\tt   (pp3d\_localpath (*)(p3d\_rob*, configPt,
  configPt))(p3d\_linear\_localplanner), }\\
\hsp{\tt   (pp3d\_localpath (*)(p3d\_rob*, configPt,
  configPt))(p3d\_rs\_localplanner), }\\
\hsp{\tt   (pp3d\_localpath (*)(p3d\_rob*, configPt,
  configPt))(p3d\_arm\_localplanner), }\\
\hsp{\tt   (pp3d\_localpath (*)(p3d\_rob*, configPt,
  configPt))(p3d\_manh\_localplanner) }\\
\hsp{\tt   (pp3d\_localpath (*)(p3d\_rob*, configPt,
  configPt))(p3d\_trailer\_localplanner) }\\
\hsp{\tt   (pp3d\_localpath (*)(p3d\_rob*, configPt,
  configPt))(p3d\_newlp\_localplanner) }\\
$\}${\tt ;}\\
\end{tabular}

\noindent
Add the name of the new local planner in {\tt
  array\_localplanner\_name[]}\index{array\_localplanner\_name}


\begin{tabular}{l}
{\tt         char * array\_localplanner\_name[] = }\\
$\{$ \\
\hsp{\tt           "RS+linear", }\\
\hsp{\tt           "Linear", }\\
\hsp{\tt           "Manhattan", }\\
\hsp{\tt           "trailer" }\\
\hsp{\tt           "NewLPname" }\\
$\}${\tt ; }\\
\end{tabular}

\noindent
Update the number of local planners

\begin{tabular}{l}
{\tt int P3D\_NB\_LOCAL\_PLANNER = 5; }
\index{P3D\_NB\_LOCAL\_PLANNER}\\
\end{tabular}

\noindent
In a new file {\tt localpath/p3d\_newlp.c}, write all the 
functions specific to the new local method and enumerated in
Section~\ref{sec:representation}. Write a function that allocates a
local path of this new type.

\begin{tabular}{l}
{\tt p3d\_localpath * p3d\_alloc\_newlp\_localpath(...) }\\
\end{tabular}

\noindent
This function has to initialize the fields {\tt length\_lp} and {\tt range\_param}.


\noindent
Add the name of this file in Init.make.move3d:

\begin{tabular}{l}
{\tt SRC\_LOCALPATH = $\backslash$ }\\
{\tt rs\_dist.c $\backslash$ }\\
{\tt rs\_curve.c $\backslash$ }\\
{\tt p3d\_local.c $\backslash$ }\\
{\tt p3d\_linear.c $\backslash$ }\\
{\tt p3d\_reeds\_shepp.c $\backslash$ }\\
{\tt p3d\_manhattan.c $\backslash$ }\\
{\tt p3d\_trailer.c $\backslash$ }\\
{\tt p3d\_newlp.c }\\
\end{tabular}
