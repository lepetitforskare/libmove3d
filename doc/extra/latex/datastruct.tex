\chapter{Data structures}
\label{datastruct}

This annex presents the different data structures used in Move3D.

\begin{itemize}

\item[$\bullet$]{\bf Environment structure : p3d\_env} \index{Data structures!p3d\_env}\\

typedef struct env \{ \\ 
\begin{tabular}{l l l}
  char      & *name; & name of the environnement\\
  int       & num; & number of this environnement in the Move3D
session \\
            &      & (the user can load several scenes)\\
  int       & no; & number of obstacles in this environement\\
  int       & nr; & number of robots in this environement \\
  p3d\_obj  & **o; & array of the obstacles of this environement\\
  p3d\_obj  & *ocur; & current obstacle of this environement\\
  p3d\_rob  & **r; & array of the robots of this environement\\
  p3d\_rob  & *rcur;& current robot of this environement\\
  p3d\_box  & box; & bounding box of this environment\\
  struct graph & *graph; & current roadmap for the current robot\\ 
\end{tabular}\\
\} p3d\_env,*pp3d\_env; \\

\item[$\bullet$]{\bf Obstacle/Body structure : p3d\_obj} \index{Data structures!p3d\_obj}\\

typedef struct obj \{\\
\begin{tabular}{l l l}
  char       & *name; & name of the objet\\
  int        &  num; &  number of the obstacle in the scene \\
             &       &  or of the body in the robot\\
  int        &  type; &  P3D\_BODY or P3D\_OBSTACLE\\
  struct env & *env; &  environnement in which the object is\\
  int np;    &        &  number of polyhedrons of the objetc\\   
  p3d\_poly   & **pol &  array of polyhedrons of the object\\
  p3d\_poly   & *polcur; &  current polyhedron of the object\\    
%  p3d_box    & box; &  \\
  struct jnt & *jnt; & joint that commands the body\\ 
             &       & (NULL is the object is an obstacle)\\
  p3d\_BB     &  BB0; & bounding box of the object \\
\end{tabular}\\
\} p3d\_obj, *pp3d\_obj;\\


\item[$\bullet$]{\bf Robot structure : p3d\_rob} \index{Data structures!p3d\_rob}\\

typedef struct rob \{\\
\begin{tabular}{l l l}
  char       & *name; & name of the robot \\
  int        & num;   &  number of the robot in the environement\\
  struct env & *env; & environment to which the robot belongs\\
  double     & radius;  &  radius of the turning circle of the robot \\
  int        & no;     &  number of bodies of the robot \\
  p3d\_obj    & **o; & array of the bodies of the robot\\
  p3d\_obj    & *ocur; &  current body of the robot\\
  int\        & njoints; &  number of joints of the robot \\
  p3d\_jnt    & **joints; & array of the joints of the robot \\
  p3d\_jnt    & *j\_modif; & first joint in the cinamtic chain that\\
              &            & has been lately modified \\
  int        & nt;    &  number of trajectories of the robot \\
             &        &  stored during the session \\
  p3d\_trj    & **t; & array of the trajectories of the robot \\
  p3d\_obj    & *tcur; &  current trajectory of the robot\\
  p3d\_box    & box; & bounds of j0\\
  int        & coll;   & boolean indicating if the robot is in\\
             &         & collision in is current position \\
  p3d\_BB     & BB; & bounding box of the robot\\
  double & *ROBOT\_POS;  &   current position \\
  double & *ROBOT\_GOTO; &   goal position \\
\end{tabular}\\
\} p3d\_rob,*pp3d\_rob;\\



\item[$\bullet$]{\bf Joint structure : p3d\_jnt} \index{Data structures!p3d\_jnt}\\

typedef struct jnt \{\\
\begin{tabular}{l l l}
  int          &  type; &  P3D\_ROTATE (rotoid joint) or\\
               &        &  P3D\_TRANSLATE (prismatic joint)\\
  int          &  num; &  number of the joint in the cinematic chain\\
  struct rob   &  *rob; &  robots to which belongs the joint\\
  struct obj   &  *o; & body attached to this joint \\
  double    &  v;   &    current value of the joint \\
  double    &  vmin; &  minimum bound of the joint \\
  double    &  vmax; & maximum bound of the joint\\
  p3d\_vector3    &  axe;  & axis of rotation or translation of the joint \\
  p3d\_point    &  p0;  & point where the joint is attached on the
previous body \\
  p3d\_matrix4  &  pos;  & matrix of position of the joint \\
                &        & (influenced by the previous joints in the chain)\\
  struct jnt   &  *prev\_jnt; & previous joint in the chain\\
               &              & (father joint)\\
  struct jnt   &  **next\_jnt; & next joints in the chain\\
               &        \     & (children joints)\\
  int          &  n\_next\_jnt; & number of joints steming from this
joint \\
  double       &  dist; & distance from p0 to the furthest point of
the body \\
\end{tabular}\\
\}p3d\_jnt, *pp3d\_jnt;\\

\item[$\bullet$]{\bf Polyhedron structure : p3d\_poly} \index{Data structures!p3d\_poly}\\

typedef struct p3d\_poly \{\\
\begin{tabular}{l l l}
  p3d\_polyhedre & *poly; &  geometrical structure of the polyhedron \\
  int & id;       &  number of the polyhedron in the scene \\
  p3d\_box & box; & bounding box\\
  p3d\_matrix4 & pos0; & matrix of the position \\
  int & color; & color of the polyhedron \\
  int & MODIF; & boolean indicating if the polyhedron is being
created\\
\end{tabular}\\
\} p3d\_poly;\\

\item[$\bullet$]{\bf Trajectory structure : p3d\_traj} \index{Data structures!p3d\_traj}\\

typedef struct traj \{\\
\begin{tabular}{l l l}
  char       & *name; & name of the trajectory \\
  char       & *file; & filename the trajectory is saved in \\
  int        & num; & index of the trajectory in the scene\\
  int        & nloc; & number of local paths of this trajectory\\
  struct rob & *rob; & robot that follow this trajectory\\ 
  double     & range\_param; & range of parameter along the trajectory \\
  p3d\_localpath & *courbePt; & first element of the list of local paths \\
\end{tabular}\\
\} p3d\_traj,*pp3d\_traj;\\

\item[$\bullet$]{\bf Local path structure : p3d\_localpath} \index{Data structures!p3d\_localpath}\\

typedef struct localpath\{\\
\begin{tabular}{l}\\
  generic and specific data \\
\end{tabular}\\
\begin{tabular}{l l l}
  p3d\_localpath\_type &    type\_lp; &  type of local path \\
  p3d\_lm\_specific &specific; & pointer to data specific to each type
  of local path \\
  struct localpath &*prev\_lp; &  the local paths can be put in a list\\
  struct localpath &*next\_lp; &\\
  int  & valid; &            FALSE if collision or if local planner
  failed to return a valid path \\
  int  & lp\_id;&             index of loc path in a trajectory \\ 
  double &  length\_lp;&       length of local path \\
  double & range\_param;&      parameter range: [0,range\_param] \\
\end{tabular}\\
\begin{tabular}{l}\\
   methods associated to the local path \\
\\
  compute length of local path 
\end{tabular}\\
\begin{tabular}{l l l}
  double & (*length)(& struct rob *, struct localpath*); \\
\end{tabular}\\
\begin{tabular}{l}\\
  copy the local path 
\end{tabular}\\
\begin{tabular}{l l l}
  struct localpath* &(*copy)(&struct rob*, struct localpath*); \\
\end{tabular}\\
\begin{tabular}{l}\\
  extract from a local path a sub local path starting from length
  l1 and ending at length l2 
\end{tabular}\\
\begin{tabular}{l l l}
  struct localpath* &(*extract\_sub\_localpath)(&struct rob *, \\
                                             &&struct localpath *localpathPt, \\
                                             &&double l1, double l2);\\
\end{tabular}\\
\begin{tabular}{l}\\
   destroy the localpath 
\end{tabular}\\
\begin{tabular}{l l l}
  void &(*destroy)(&struct rob*, struct localpath*); \\
\end{tabular}\\
\begin{tabular}{l}\\
  compute the configuration at given distance along the path 
\end{tabular}\\
\begin{tabular}{l l l}
  configPt &(*config\_at\_distance)(&struct rob*, \\
                                 &&struct localpath*, \\
                                 &&double); \\
\end{tabular}\\
\begin{tabular}{l}\\
   compute the configuration for a given parameter along the path
\end{tabular}\\
\begin{tabular}{l l l}
  configPt &(*config\_at\_param)(&struct rob*,\\
                              &&struct localpath*,\\
                              &&double); \\
\end{tabular}\\
\begin{tabular}{l}\\
from a configuration on a local path, this function computes an \\
interval of parameter on the local path on which all the points \\
of the robot move by less than the distance given as input.\\
The interval is centered on the configuration given as input. The \\
function returns the half length of the interval 
\end{tabular}\\
\begin{tabular}{l l l}
  double &(*stay\_within\_dist)(&struct rob*,\\
                             &&struct localpath*,\\
                             &&double, whichway, double*);\\
\end{tabular}\\
\begin{tabular}{l}\\
  compute the cost of a local path 
\end{tabular}\\
\begin{tabular}{l l l}
  double &(*cost)(&struct rob*, struct localpath*); \\
\end{tabular}\\
\begin{tabular}{l}\\
  function that simplifies the sequence of two local paths: valid\\
  only for RS curves 
\end{tabular}\\
\begin{tabular}{l l l}
  struct localpath*& (*simplify)(&struct rob*,\\
                                &&struct localpath*,\\
                                &&int*); \\
\end{tabular}\\
\begin{tabular}{l}\\
\} p3d\_local\_path , *pp3d\_local\_path;\\
\end{tabular}\\

\item[$\bullet$]{\bf Pointer to data specific to a type of local path : p3d\_lm\_specific} 
\index{Data structures!p3d\_lm\_specific}\\

typedef union lm\_specific \{\\
\begin{tabular}{lll}
  pp3d\_rs\_data & rs\_data; & pointer to Reeds and Shepp local path data \\
  pp3d\_lin\_data &lin\_data; & pointer to linear local path data \\
  pp3d\_manh\_data &manh\_data;& pointer to Manhattan local path data \\
  pp3d\_trailer\_data &trailer\_data; & pointer to trailer local path
  data \\
\end{tabular}\\
\} p3d\_lm\_specific, *pp3d\_lm\_specific;

\item[$\bullet$]{\bf Linear local path structure : p3d\_lin\_data} 
\index{Data structures!p3d\_lin\_data}\\

Specific data relative to a linear local path\\
typedef struct lin\_data \{\\
\begin{tabular}{lll}
  configPt &q\_init;     &config init sur la courbe   \\
  configPt &q\_end;       &config fin  sur la courbe  \\
\end{tabular}\\
\} p3d\_lin\_data, *pp3d\_lin\_data;\\


\item[$\bullet$]{\bf Manhattan local path structure : p3d\_manh\_data} 
\index{Data structures!p3d\_manh\_data}\\

Specific data relative to a Manhattan local path\\
typedef struct manh\_data\{\\
\begin{tabular}{lll}
  configPt &q\_init;     &initial configuration   \\
  configPt &q\_end;       &final configuration  \\
\end{tabular}\\
\} p3d\_manh\_data, *pp3d\_manh\_data;\\


\item[$\bullet$]{\bf Reeds and Shepp local path structure : p3d\_rs\_data} 
\index{Data structures!p3d\_rs\_data}\\

Specific data relative to a Reeds and Shepp local path\\
typedef struct rs\_data\{\\
\begin{tabular}{lll}
  configPt &q\_init;    &initial configuration \\
  configPt &q\_end;       &final configuration  \\
  double &centre\_x;      &abscissa of the center of rotation if any \\
  double &centre\_y;       &ordinate of the center of rotation if any \\
  double &radius;          &radius of curvature \\
  whichway &dir\_rs;        &direction in \{BACKWARD, FORWARD\} \\
  double &val\_rs;          &length of RS segment \\
  rs\_type &type\_rs;        &type = 1/2/3 for right/left/straight \\
  struct rs\_data &*next\_rs;  &next RS segment in localpath \\ 
  struct rs\_data &*prev\_rs;  &previous RS segment in localpath \\ 
\end{tabular}\\
\} p3d\_rs\_data, *pp3d\_rs\_data;


\item[$\bullet$]{\bf trailer local path structure : p3d\_trailer\_data
    and  p3d\_sub\_trailer\_data}\\
\index{Data structures!p3d\_trailer\_data}
\index{Data structures!p3d\_sub\_trailer\_data}

Specific data relative to a trailer local path\\
typedef struct trailer\_data\{\\
\begin{tabular}{lll}
  p3d\_sub\_trailer\_data &*init\_cusp; & first part of localpath from
  q\_init to a cusp configuration \\
  p3d\_sub\_trailer\_data &*cusp\_end; &second part of localpath from
  q\_cusp to q\_end\\
\end{tabular}\\
\} p3d\_trailer\_data, *pp3d\_trailer\_data;\\


typedef struct sub\_trailer\_data\{\\
\begin{tabular}{lll}
  configPt &q\_init; & init config on sub local path   \\
  configPt &q\_end; & end config on sub local path   \\
  double  &v; & projection of q\_end on Gamma\_init \\
  double &length;    &length of the sub local path between q\_init and q\_end\\
  double &alpha\_0;    & third derivative of alpha at beginning \\
  double &alpha\_1;    & third derivative of alpha at end \\
  double & u\_start;   & parameter beginning of localpath on combination \\
  double & u\_end;     & parameter end of localpath on combination \\
  double &gamma\_1\_min; &the min of the first derivation of the curve\\
  double &V\_1\_rob\_max;&acceleration of all point of the tractor on all the sub path\\
  double &V\_1\_rem\_max;&acceleration of all point of the trail on all the sub path\\
  double &phi\_max;&the max of phi on all the way\\
  double &phi\_min;&the min of phi on all the way\\
  double &phi\_1\_tot;&this is the kind of integral of phi\_2 on the
  path\\
\end{tabular}\\
\}p3d\_sub\_trailer\_data, *pp3d\_sub\_trailer\_data; 


\item[$\bullet$]{\bf Graph structure : p3d\_graph} 
\index{Data structures!p3d\_graph}\\

typedef struct graph \{\\
\begin{tabular}{l l l}
  p3d\_env & *env; & environment corresponding to the graph \\
  p3d\_rob & *rob; & robot corresponding to the graph \\
  int & nnode; & number of nodes of the graph \\
  p3d\_list\_node & *nodes; & list of the nodes of the graph \\
  p3d\_list\_node & *last\_node; & last node of the graph created \\
  int & nedge; & number of edges of the graph \\
  p3d\_list\_edge & *edges; & list of the edges of the graph \\
  p3d\_list\_edge & *last\_edge; &  last edge of the graph created\\
  int & ncomp; & number of connected components of the graph \\
  p3d\_compco & *comp; & list of the connected components of the graph \\
  p3d\_compco & *last\_comp; & last connected component of the graph created \\
              &              &     \\
  p3d\_node   & *search\_start; & start node for the graph search\\
  p3d\_node   & *search\_goal;  & goal node for the graph search\\
  p3d\_node   & *search\_goal\_sol; & goal node found after search\\
  int        & search\_done; & search done ? T/F\\
  int        & search\_path; & is there a path ? T/F\\
  int        & search\_numdev; & number of nodes searched\\
  double     & search\_cost; & cost of the path found\\
  int        & search\_path\_length; & number of edges of the path found\\
\end{tabular}\\
\begin{tabular}{l l l}
  double & time; & time needed to build the graph\\
  int & nb\_test\_BB; & number of bounding box tests performed\\
  int & nb\_test\_coll; & number of collision checking tests performed\\
  int & nb\_local\_call; & number of calls to the local method\\
  int & nb\_q; & total number of random configurations generated\\
  int & nb\_q\_free; &  total number of free random configurations generated\\
\end{tabular}\\
\} p3d\_graph, *pp3d\_graph;\\

\item[$\bullet$]{\bf Node structure : p3d\_node} \index{Data structures!p3d\_node}\\

typedef struct node \{\\
\begin{tabular}{l l l}
  int         & type; & type of the node (isolated, linking)\\
  int         & num ;& number of the node in the graph\\
  int         & numcomp; & number of the connected component \\
              &          & to which the node belongs\\
  struct compco & *comp; & connected component to which the node belongs\\
  double      & *q; & configuration of the robot for this node\\
  int & nneighb; &    number of neighbours of the node\\
  struct list\_node & *neighb; & list of nighbours of the node\\
  struct list\_node & *last\_neighb; & last neighbour of the node\\
  int & nedge;  & number of edges attached to the node\\
  struct list\_edge & *edges;  & list of the edges attached to the node\\
  struct list\_edge & *last\_edge; & last edge attached to the node\\
  double & dist\_Nnew; & distance to the node being created\\
         &            &    \\
  double & f; &   value of the heuristic function for this node + \\
         &    &   lenght of the current path\\                               
  double & g; &   lenght of the current arc coming to this node\\
  double & h; &   value of the heuristic function for this node\\
  short & opened; & has the node already been seen ? T/F\\
  short & closed; & has the node already been seen ? T/F\\
  struct node & *search\_from; & node having lead to this node\\
  struct node & *search\_to;  & neighbour node kept in the search\\
 \end{tabular}\\
\} p3d\_node, *pp3d\_node;\\

\item[$\bullet$]{\bf List of nodes structure : p3d\_list\_node}
\index{Data structures!p3d\_list\_node}\\

typedef struct list\_node\{\\
\begin{tabular}{l l l}
  p3d\_node & *N; & node contained in this part of list\\
  struct list\_node & *next; & next node in the list\\
  struct list\_node & *prev; & previous node in the list\\
\end{tabular}\\
\} p3d\_list\_node;\\

\item[$\bullet$]{\bf Edge structure : p3d\_edge}
\index{Data structures!p3d\_edge}\\

typedef struct edge \{\\
\begin{tabular}{l l l}
  p3d\_node  & *Ni; & initial node of the edge\\
  p3d\_node  & *Nf; & final node of the edge\\
  int        & planner; & local method used for this edge\\
  double     & longueur; & lenght of this edge\\
  int & sens; & direction of the edge\\
\end{tabular}\\
\} p3d\_edge;\\


\item[$\bullet$]{\bf List of edges structure : p3d\_list\_edge}
\index{Data structures!p3d\_list\_edge}\\

typedef struct list\_edge\{\\
\begin{tabular}{l l l}
  p3d\_node & *E; & edge contained in this part of list\\
  struct list\_node & *next; & next edge in the list\\
  struct list\_node & *prev; & previous edge in the list\\
\end{tabular}\\
\} p3d\_list\_edge;\\

\item[$\bullet$]{\bf Connected component structure : p3d\_compco}
\index{Data structures!p3d\_compco}\\

typedef struct compco \{\\
\begin{tabular}{l l l}
  int & num; & number of the conected component\\
  int & nnode; & number of nodes of the conected component\\
  char & *nodes; & list of the nodes of the connected component for the \\
       &         & sorting regarding the number of edges of
each node\\
  p3d\_list\_node & *dist\_nodes; &list of the nodes of the connected
component for the\\
       &         & sorting regarding the distance to th node
being created\\
  p3d\_list\_node & *last\_node; & last node of the connected component\\
  struct compco & *suiv; & next conected componant\\
  struct compco & *prec; & previous connected component\\
\end{tabular}\\
\} p3d\_compco;\\


\item[$\bullet$]{\bf Box structure : p3d\_box} \index{Data structures!p3d\_box}\\

typedef struct box \{\\
\begin{tabular}{l l l}
  double & x1,x2,y1,y2,z1,z2; & bounds of the box \\
\end{tabular}\\
\} p3d\_box, *pp3d\_box;\\

\item[$\bullet$]{\bf Point structure : p3d\_point} \index{Data structures!p3d\_point}\\

typedef struct p3d\_point \{\\
\begin{tabular}{l l l}
  double & x, y, z; & 3D coordinates of the point\\
\end{tabular}\\
\} p3d\_point;\\

\item[$\bullet$]{\bf Bounding box structure : p3d\_BB} \index{Data structures!p3d\_BB}\\

typedef struct BB \{ \\ 
\begin{tabular}{l l l}
    double & xmin,ymin,zmin; & bounds of the box\\
    double & xmax,ymax,zmax; & \\
\end{tabular}\\
\} p3d\_BB, *pp3d\_BB;

\end{itemize}
