\chapter{Motion planning}
\label{global}

\section{Shooting a random configuration}

Move3D is dedicated to PRM-like probabilistic motion planning
(\cite{KAVRA95,KSLO96,SVET97}). The basic function of a probabilistic
motion planner is shooting a free random configuration in the configuration
space of the robot. Move3D proposes several function to generate
free random configurations.

The function {\tt void p3d\_init\_random\_seed(int seed)}
\index{p3d\_init\_random\_seed} initializes
the C random function with the integer {\tt seed}. This function
allows the user to shoot the same sequence of random numbers several
times. The function {\tt void p3d\_init\_random(void)}
\index{p3d\_init\_random} initializes
the C random function with a value depending of the clock of the
computer. This function allows the user to get random sequences ``really
shot at random''.

The function {\tt double p3d\_random(double a, double b)}
\index{p3d\_random} returns a
random value between {\tt a} and {\tt b}. 
The function {\tt void p3d\_shoot(p3d\_rob *r, double *q)}
\index{p3d\_shoot} shoots a
configuration in the configuration space of the robot {\tt r}. The
function {\tt double *p3d\_shoot \_graph(p3d\_rob *r, p3d\_graph *G)}
\index{p3d\_shoot\_graph} does
exactly the same but returns a free configuration for the robot {\tt
r} (and updates the configurations counter of the graph {\tt G}).

The function {\tt double *p3d\_shoot\_walk(p3d\_rob *r, double *q,
double *delta)} \index{p3d\_shoot\_walk} shoots, for the robot {\tt r} with n degrees of
freedom, a configuration distant from a step {\tt dq} of the
configuration {\tt q} : n values v equal to 1, 0 or -1 are shot and the
new configuration is obtained by adding {\tt q}[i]+ v*{\tt delta}[i], for
i from 1 to n.

The function {\tt void p3d\_shoot\_in\_box(p3d\_rob *r, double *q,
double *qn, double *dq)} \index{p3d\_shoot\_in\_box} shoots a configuration in the box of the
configuration space centered in {\tt qn} and of amplitude {\tt dq}. Th
function {\tt double *p3d\_shoot\_box(p3d\_rob *r, p3d\_graph *G,
double *qn, double *dq)} \index{p3d\_shoot\_box} does exactly the same but returns a free
configuration for the robot {\tt r} (and updates the configurations
counter of the graph {\tt G}).

\section{Building a graph}

The function {\tt p3d\_graph *p3d\_create\_graph(void)}
\index{p3d\_create\_graph} allocates and
initializes a graph structure {\tt p3d\_graph} (cf Annex \ref{datastruct}), stores it
as the current graph, and returns it as an output.

The function {\tt p3d\_add\_node(p3d\_graph *G, double *q)}
\index{p3d\_add\_node} allocates
a node structure {\tt p3d\_node} (cf Annex \ref{datastruct})
corresponding to the configuration {\tt q}. This function initializes
the new node, adds it to the graph {\tt G} and updates the number of
nodes of this graph. It also stores this node as the current node and
returns it as an output.

The function {\tt void p3d\_create\_compco(p3d\_graph *G, p3d\_node
*N)} \index{p3d\_create\_compco} creates a new connected component of the graph {\tt G} containing
the node {\tt N}. It allocates a connected componant structure {\tt
p3d\_compco} (cf Annex \ref{datastruct}), initializes it, adds it to
the graph {\tt G} and updates the number of connected componant of this
graph. 

The function {\tt void p3d\_create\_edges(p3d\_rob *rob, p3d\_graph
*G, p3d\_node *N1, p3d\_node *N2, double dist)}
\index{p3d\_create\_edges} adds the edges
[{\tt N1,N2}] and [{\tt N2,N1}] of lenght {\tt dist} to the graph {\tt G} of the robot {\tt rob}
(Move3D only deals with symmetrical local methods so when the edge
[{\tt N1,N2}] exists, the edge [{\tt N2,N1}] exists too). This function
allocates two edges structures {\tt p3d\_edges} (cf Annex
\ref{datastruct}), initializes them (with N1 et N2 as end nodes),
adds them to the graph, as egdes coming from and going to {\tt N1} and
{\tt N2} and updates the number of edges of the graph {\tt G}.

The function {\tt void p3d\_add\_neighbour(p3d\_graph *G, p3d\_node
*N1, p3d\_node *N2)} \index{p3d\_add\_neighbour} adds the node {\tt
N2} to the list of neighbours of the node {\tt N1}.

The function {\tt int p3d\_linked(p3d\_rob *rob, p3d\_graph *G, p3d\_node
*N1, p3d \_node *N2, double *dist)} \index{p3d\_linked} indicates if the nodes {\tt N1} and
{\tt N2} are linked by a collision free path built by the current
local method of the robot {\tt rob}. The function {\tt int
p3d\_link\_node\_comp(p3d\_rob *rob,p3d\_graph *G, p3d\_node
*N,p3d\_compco *comp)} \index{p3d\_link\_node\_comp} does exactly the
same for the node {\tt N} and the connected componant {\tt comp}.

The function {\tt void p3d\_merge(p3d\_graph *G,p3d\_compco *c1,
p3d\_compco *c2)} \index{p3d\_merge} merges the connected componant {\tt c2} of the graph
{\tt G} in the connected componant {\tt c1} of {\tt G}. 


\section{Existing motion planners}

Two types of motion planners have been implemented in Move3D, using
two different strategies. 

The function {\tt void p3d\_learn(in NMAX)} \index{p3d\_learn} builts a roadmap for the
current robot in the current environnement, using the current planning
strategy, but without trying to connect the current and goal
configurations of the robot. The function {\tt int
p3d\_specific \_learn(double *qs, double *qg)}
\index{p3d\_specific\_learn} builts the same type of
graph until the configurations {\tt qs} and {\tt qg} are in the same
connected component. 

Of course, those two motion planners need a termination criterion, to
decide if a roadmap is sufficient or if the two configurations can not
be connected. This criterion depends on the planning strategy chosen.

Th first strategy, that will be called basic strategy, is the one
described in \cite{SVET97}. Each free configuration generated is added
to the graph and linked to one or several connected componants. 
The function {\tt int p3d\_add\_basic\_node(p3d\_graph *G, p3d\_rob
*rob, int inode, int SHOOT, double *qn, double *dq)} \index{p3d\_add\_basic\_node} adds a node to
the graph {\tt G} of the robot {\tt rob} (shot in the whol
configuration space or in the box centered in {\tt qn} and of
amplitude {\tt dq}, depending on the value of {\tt SHOOT}) according to
this motion strategy. In this case, the termination criterion is the
maximum number of nodes allowed for the graph {\tt NMAX}. 

The second strategy, that will be called visibility based strategy, is
the one described in \cite{NISS99}. A free configuration is added to
the graph only if it does not ``se{e}'' any connected component of the
graph (if it is not linked to any node of any connected component), it
will then become a warden node, or if it sees at least two connected
component, it will then become a connection node. The function {\tt int
p3d\_add\_isolate\_or\_linking\_node(p3d\_graph *G, p3d\_rob *rob, int
inode, int  *fail, int SHOOT, double *qn, double *dq)}
\index{p3d\_add\_isolate\_or\_linking\_node} adds a node to
the graph {\tt G} of the robot {\tt rob} according to this motion
strategy. In this case, the termination criterion is the maximum
number of configuration generated that did not become a warden node
allowed. This number of failure to find a new warden node reflects the
covering of the graph (cf \cite{NISS99}).

\section{Expanding a graph}

Once a graph has been built, the user can decide to expand it in order
to improve its covering or its connectivity. 

The function {\tt void p3d\_expand\_graph(p3d\_graph *G, double frac)}
\index{p3d\_expand\_graph}
searches {\tt G} for all the connected componant whose nodes represents less
than {\tt frac} \% of the number of nodes of {\tt G} and expands all
the nodes of thoses connected componants, according to the expansion
strategy chosen, box expansion or random walk expansion. 

Th function {\tt int p3d\_expand\_box(p3d\_node *N, double *dq, int
NB\_NODE, p3d \_graph *G)} \index{p3d\_expand\_box} creates {\tt NB\_NODE} extra nodes in the box
centered in the configuration of the node {\tt N} of amplitude {\tt
dq} and adds them to the graph {\tt G} according to the planning strategy
chosen. 

The function {\tt p3d\_expand\_random\_walk(p3d\_node *N, double
*delta, int nbstep, int NB\_WALK, p3d\_graph *G)}
\index{p3d\_expand\_random\_walk} performs {\tt NB\_WALK}
random walks of {\tt nbstep} steps of amplitude {\tt delta} from the
node {\tt N} and adds them to the graph if the path created is valid.

\section{Searching a graph}

Once a graph has been created, we need a function that can search it
to see if two nodes are connected : the function {\tt int
p3d\_graph\_search(p3d\_graph *graph,double (*fct\_heurist)(),int
(*fct\_valid)(),int (*fct\_end)())} \index{p3d\_graph\_search} that runs a A* algorithm. This
function checks if the nodes {\tt graph->search\_start} and {\tt
graph->search\_goal} are connected in the graph (cf Annex
\ref{datastruct}), with the function {\tt fct\_heurist} being the
heuristic for the search (usually the function {\tt double
p3d\_heurist(p3d\_node *n1, p3d\_node *n2, p3d\_graph *G)}), the
function {\tt fct\_valid} being a checking function (usually the
function {\tt int p3d\_valid(p3d\_edge *E, p3d\_graph *G)}) and the
function {\tt fct\_end} being the termination criterion function
(usually the function {\tt int p3d\_end(p3d\_node *n1, p3d\_node
*n2)}).

The list of edges connecting {\tt graph->search\_start} and {\tt
graph->search\_goal} allows the user to build the trajectory
connecting the two configuration with the sequence of valid local
paths corresponding to those edges, using the functions {\tt
p3d\_beg\_traj, p3d\_add\_desc\_courbe, p3d\_end\_traj} and the
current local method.

\section{Optimizing a trajectory}

Once a trajectory has been built, the user can decide to optimize it,
to reduce its lenght. The function {\tt int p3d\_optim(p3d\_trj *t,
double *gain)} \index{p3d\_optim} optimize the trajectory {\tt t} by dividing it randomly
in three parts and replacing any of those parts that can be
replaced by a valid local path. The gain of lenght obtained is
returned.   

\section{Deleting a graph}

A graph can be entierly deleted with the function {\tt int
p3d\_del\_graph(p3d\_graph *G)} \index{p3d\_del\_graph}, that liberates all the nodes,
connected components and edges, and the graph, but the user can
also decide to delete only one node at the time withe the function 
{\tt void p3d\_del\_node(p3d\_node *N, p3d\_graph *G)}
\index{p3d\_del\_node}. This function deletes the node {\tt N} and
updates the structure of the graph {\tt G}.
