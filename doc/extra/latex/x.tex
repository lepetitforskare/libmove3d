\relax 
\@writefile{toc}{\contentsline {chapter}{\numberline {\rmfamily  4}Kinematic Constraints}{39}}
\@writefile{lof}{\addvspace {10pt}}
\@writefile{lot}{\addvspace {10pt}}
\newlabel{Constraints}{{\rmfamily  4}{39}}
\@writefile{toc}{\contentsline {section}{\numberline {\rmfamily  1}Introduction}{39}}
\@writefile{toc}{\contentsline {section}{\numberline {\rmfamily  2}Method}{39}}
\@writefile{toc}{\contentsline {section}{\numberline {\rmfamily  3}Treated constraints}{40}}
\@writefile{lof}{\contentsline {figure}{\numberline {\rmfamily  4.1}{\ignorespaces  Two holonomic robots cooperating in a task. The bar with the weighs must stay horizontal, so the hydraulic bars of the robots must move at the same time.}}{41}}
\newlabel{fig:halteres}{{\rmfamily  4.1}{41}}
\@writefile{lof}{\contentsline {figure}{\numberline {\rmfamily  4.2}{\ignorespaces  Constraint in the movement of a finger. The left figure shows an abnormal position for a finger, the right one shows a normal situation.}}{42}}
\newlabel{fig:dedos}{{\rmfamily  4.2}{42}}
\@writefile{lof}{\contentsline {figure}{\numberline {\rmfamily  4.3}{\ignorespaces  The dimensions characterizing an $RRPR$ linkage.}}{42}}
\newlabel{fig:RRPR}{{\rmfamily  4.3}{42}}
\@writefile{lof}{\contentsline {figure}{\numberline {\rmfamily  4.4}{\ignorespaces  The dimensions characterizing an $4R$ linkage.}}{43}}
\newlabel{fig:4R}{{\rmfamily  4.4}{43}}
\@writefile{lof}{\contentsline {figure}{\numberline {\rmfamily  4.5}{\ignorespaces  The dimensions characterizing an $3RPR$ linkage.}}{44}}
\newlabel{fig:3RPR}{{\rmfamily  4.5}{44}}
\@writefile{lof}{\contentsline {figure}{\numberline {\rmfamily  4.6}{\ignorespaces  Hydraulic excavator. This model contains two single closed chains and one double loop closed chain. The arm, a mechanical system of 19 d.o.f. has been constrained to a system of 3 d.o.f..}}{45}}
\newlabel{fig:excavator}{{\rmfamily  4.6}{45}}
\@writefile{lof}{\contentsline {figure}{\numberline {\rmfamily  4.7}{\ignorespaces  Manipulator shown in some works of Liegeois. This two d.o.f. mechanical system consist of a double loop closed chain.}}{46}}
\newlabel{fig:liegeois}{{\rmfamily  4.7}{46}}
\@writefile{lof}{\contentsline {figure}{\numberline {\rmfamily  4.8}{\ignorespaces  Variables and references in the analysis.}}{46}}
\newlabel{fig:ground}{{\rmfamily  4.8}{46}}
\@writefile{lof}{\contentsline {figure}{\numberline {\rmfamily  4.9}{\ignorespaces  A crane has to place a tank into a metallic structure in an industrial environment. In the initial configuration, the tank is horizontally placed on the ground. The tank has to slide on the ground before it reaches the vertical position.}}{47}}
\newlabel{fig:gruaind}{{\rmfamily  4.9}{47}}
\@writefile{lof}{\contentsline {figure}{\numberline {\rmfamily  4.10}{\ignorespaces  A telescopic handler carries a long pipe. The lowest extreme of the pipe must stay in contact with the ground all along the path.}}{48}}
\newlabel{fig:pipe}{{\rmfamily  4.10}{48}}
\@writefile{lof}{\contentsline {figure}{\numberline {\rmfamily  4.11}{\ignorespaces  Geometric law for the angles of the car front wheels.}}{48}}
\newlabel{fig:carfw}{{\rmfamily  4.11}{48}}
\@writefile{lof}{\contentsline {figure}{\numberline {\rmfamily  4.12}{\ignorespaces  Three-dimensional model of a LegoCar.}}{49}}
\newlabel{fig:LegoCar}{{\rmfamily  4.12}{49}}
\@writefile{lof}{\contentsline {figure}{\numberline {\rmfamily  4.13}{\ignorespaces  Variables and references in the analysis.}}{49}}
\newlabel{fig:plclch}{{\rmfamily  4.13}{49}}
\@writefile{lof}{\contentsline {figure}{\numberline {\rmfamily  4.14}{\ignorespaces  Some sequences of the solution given by Move3D for a 8-bar closed chain that has to find a path trough a narrow passage.}}{50}}
\newlabel{fig:cad8pas}{{\rmfamily  4.14}{50}}
\@writefile{toc}{\contentsline {section}{\numberline {\rmfamily  4}How to use them}{50}}
\@writefile{lof}{\contentsline {figure}{\numberline {\rmfamily  4.15}{\ignorespaces  Kinematic constraints window. In this case, the list of constraints corresponds to the model of the excavator arm (see Figure\nobreakspace  {}\rmfamily  4.6\hbox {}).}}{53}}
\newlabel{fig:kcwin}{{\rmfamily  4.15}{53}}
\@writefile{toc}{\contentsline {section}{\numberline {\rmfamily  5}Implementation}{54}}
\@writefile{lof}{\contentsline {figure}{\numberline {\rmfamily  4.16}{\ignorespaces  Kinematic constraints window, kinematic constraints setting window, window containing the parameters of a constraint and setting window of a constraint.}}{55}}
\newlabel{fig:jogwins}{{\rmfamily  4.16}{55}}
\@writefile{toc}{\contentsline {section}{\numberline {\rmfamily  6}New kinematic constraints}{57}}
\@setckpt{constraints}{
\setcounter{page}{58}
\setcounter{equation}{0}
\setcounter{enumi}{0}
\setcounter{enumii}{0}
\setcounter{enumiii}{0}
\setcounter{enumiv}{0}
\setcounter{footnote}{0}
\setcounter{mpfootnote}{0}
\setcounter{part}{0}
\setcounter{chapter}{4}
\setcounter{section}{6}
\setcounter{subsection}{0}
\setcounter{subsubsection}{0}
\setcounter{paragraph}{0}
\setcounter{subparagraph}{0}
\setcounter{figure}{16}
\setcounter{table}{0}
\setcounter{Enumi}{0}
\setcounter{question}{0}
\setcounter{Propriete}{0}
\setcounter{proposition}{0}
\setcounter{definition}{0}
\setcounter{provcounter}{0}
}
