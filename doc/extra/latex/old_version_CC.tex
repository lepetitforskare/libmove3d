\chapter{Collision detection}

Move3D has two collision checkers : I-collide  (\cite{LINMANOCH95})
and V-collide (\cite{LINMANOCH97}). Both work the same way. Being
given some polyhedrons, the user defines which pairs of polyhedron can
be in collision. Those pairs are said ``activ{e}''. Every time one or
several polyhedron move, the collision checkers makes a list of all
the pairs of polyhedron whose bounding boxes overlap. Then an
expensive but exact algorithm checks if the pairs of this list
collide. 

Move3D allows the use of the two collision checkers, the choice of the
bounding boxes and proposes some function to simplify the activation
of the pairs of polyhedron that can collide in a whole scene.

\section{Initialization}

It is clear that a collision checker can not be started if a scene has
not been described.

The function {\tt void p3d\_col\_init\_coll(void)}
\index{p3d\_col\_init\_coll} sets all the
bolean indicating that the collision checkers have initialized to
null. It must be called when the user wants to reinitialize all the
collision checkers.

To start a collision checker, the user can choose among the three
following functions :

\begin{itemize}
\item {\tt void p3d\_col\_start\_current(void)}
\index{p3d\_col\_start\_current} will initialize
the collision checker whose type has been stored as an integer in the global
variable {\tt p3d\_col\_mode} : {\tt p3d\_col \_mode\_i\_collide} for
I-collide and {\tt p3d\_col\_mode\_v\_collide} for V-collide.
\item {\tt void p3d\_col\_start\_last(void)}
\index{p3d\_col\_start\_last} will initialize
the collision checker whose type has been stored as an integer in the global
variable {\tt p3d\_col\_last\_mode}.
\item {\tt void p3d\_col\_start(int c)} \index{p3d\_col\_start} will initialize
the collision checker whose type correspond to the integer {\tt c}.
\end{itemize}

Those three function will allocate and initialize all the variables
needed for the chosen algorithm of collision detection.

If the user has chosen I-collide, he can then choose the kind of
bounding box will be used for the polyhedrons. The function {\tt int
p3d\_switch\_all\_to\_cube(void)} \index{p3d\_switch\_all\_to\_cube}
sets the bounding boxes to cuboid bounding box, lined up on the axis,
containing the polyhedrons for any orientation, whereas {\tt int
p3d\_switch\_all\_to\_DBB(void)} \index{p3d\_switch\_all\_to\_DBB} sets the bounding box to Dynamic
Bounding Boxes, also lined up on the axis, changing of size
everytime the polyhedrons change of orientation. The collision checker
V-collide only use cuboid bounding boxes. Those bounding boxes are
used in Move3D to cumpute the bounding boxes of the obstacles and robots.

Once the collision checkers have been initialized, the user must
indicate the pairs of polyhedron, among all the possible pairs, that
may collide. Three functions are vailable :

\begin{itemize}
\item {\tt void p3d\_col\_activate\_pair(p3d\_poly *obj1,p3d\_poly
*obj2)} \index{p3d\_col\_activate\_pair} will activate the pair of polyhedron {\tt (obj1,obj2)}
\item {\tt void p3d\_col\_activate\_full(p3d\_poly *obj)}
\index{p3d\_col\_activate\_full} will activate
the pairs formed by the polyhedron {\tt obj} and all the other
polyhedrons in the scene.
\item {\tt void p3d\_col\_activate\_all(void)}
\index{p3d\_col\_activate\_all} will activate all the
pairs of polyhedrons in the scene.
\end{itemize}

Of course, an active pair can also be deactivate. The three previous
activation functions have their equivalent for deactivation : 
{\tt void p3d\_col\_deactivate\_pair(p3d\_poly *obj1,p3d\_poly *obj2)}
\index{p3d\_col\_deactivate\_pair}, 
{\tt void p3d\_col\_deactivate\_full(p3d\_poly *obj)}
\index{p3d\_col\_deactivate\_full} and 
{\tt void p3d\_col\_deactivate\_all(void)}
\index{p3d\_col\_deactivate\_all}.

It is clear that for a complex scene, the user could spend days
listing those pairs. Some function are available in Move3D in order to
simplify the activation of a whole scene. The idea is simple : in a
scene, only the robots move. So the only pairs to activate are the
pairs of one polyhedron of a robot with one polyhedron of an
obstacle, the pairs of one polyhedron of a robot with one polyhedron
of another robot, and the pairs of polyhedrons of the robot that do
not belong to two adjacent bodies (so we can detect the auto-collisions).
Those tree functionality are gathered in the function {\tt int
p3d\_col\_activate\_env(void)} \index{p3d\_col\_activate\_env} that
activates all the useful pairs of the current environment. 

%{\tt
%p3d\_col\_activate\_rob\_obj\\
%p3d\_col\_activate\_rob\_rob\\
%p3d\_col\_activate\_rob\\
%p3d\_col\_activate\_robots\\
%p3d\_col\_activate\_env\\


%p3d\_col\_deactivate\_rob\_obj\\
%p3d\_col\_deactivate\_rob\_rob\\
%p3d\_col\_deactivate\_rob\\

If the user wants to change collision checker during a session, he
firts has to stop the current one. The function {\tt void
p3d\_col\_stop(void)} \index{p3d\_col\_stop} stops the collision checker whose type has been
stored as an integer in the global variable {\tt p3d\_col\_mode}, and
deletes all its datas. The function {\tt void
p3d\_col\_stop\_all(void)} \index{p3d\_col\_stop\_all} does the same,
but for all collision checkers.

Once a collision checker and the active pairs of the environment have
been initiallized, collisions can be detected at any time.

\section{Detection of a collision}

The most useful function to detect a collision is {\tt int
p3d\_col\_test(void)} \index{p3d\_col\_test}. This function checks the whole scene and
returns the number of collisions. But the user may also want to check
a specific pair. In this case, the function {\tt int
p3d\_col\_test\_pair(p3d \_poly *obj1, p3d\_poly *obj2)}
\index{p3d\_col\_test\_pair} must be called :
it checks wether the pair {\tt (obj1,obj2)} collides and returns True
or False.

Once the collisions have been checked, some informations on the
collisions that may have occured are available. 

The function {\tt int p3d\_col\_number(void)} \index{p3d\_col\_number}
returns the number of collision that were detected by the collision
checker. The function {\tt void p3d\_col\_get\_report(int ind,
p3d\_poly *p1, p3d\_poly *p2)} \index{p3d\_col\_get\_report} reports
the collision number {\tt ind} and the pair of polyhedrons that collided.


%p3d_col_get

\section{Computing a distance}

%p3d\_affich\_dist\\

The collision checker V-collide does not compute the distance between
polyhedrons of the scene. The collision checker I-collide computes it
for the polyhedrons whose bounding boxes overlap. Anyway, one can
always compute an approached distance between two objects of a scene
using their bounding boxes.

The function {\tt double p3d\_BB\_obj\_obj\_extern\_dist(p3d\_obj
*obj1, p3d\_obj *obj2)} \index{p3d\_BB\_obj\_obj\_extern\_dist}
returns the distance between the bounding boxes of {\tt obj1} and {\tt
obj2}. The function {\tt double
p3d\_calc\_robot\_BB\_extern\_dist(p3d\_rob *rob, int *obstnum)}
\index{p3d\_calc\_robot\_BB\_extern\_dist} returns the shortest
distance between the bounding boxes of the bodies of the robot and the
bounding boxe of the obstacle. The variable {\tt obstnum} contains the
number of teh closest obstacle to the robot. The function {\tt double
p3d\_calc\_body\_BB\_extern\_dist(p3d\_obj *body, int *obstnum)}
\index{p3d\_calc\_body\_BB\_extern\_dist} returns the closest distance
from the bounding box of the body {\tt body} in its current position
to the bounding boxes of the obstacles. The integer obstnum is the
closest obstacle to {\tt body}.


Knowing when the bounding boxes of a robot, a body of a robot or an
obstacle overlap can also be useful. For this purpose, the user may
call the three following functions. The function {\tt int
p3d\_BB\_overlap\_obj\_obj(p3d\_obj *obj1, p3d\_obj *obj2)}
\index{p3d\_BB\_overlap\_obj\_obj} checks if
the bounding boxes of the objects {\tt obj1} and {\tt obj2}
overlap. The functions {\tt int p3d\_BB\_overlap\_rob\_obj(p3d\_rob
*rob, p3d\_obj *obj)} \index{p3d\_BB\_overlap\_rob\_obj} and {\tt int
p3d\_BB\_overlap\_rob \_rob(p3d\_rob *rob1, p3d\_rob *rob2)}
\index{p3d\_BB\_overlap\_rob\_rob} for the robot {rob} and the object
{obj} and for the robots {\tt rob1} and {\tt rob2}.

Those functions are mostly used for the validation of trajectories
described in the following chapters.
